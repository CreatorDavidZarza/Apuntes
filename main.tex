\documentclass{article}
\usepackage{amsmath}
\usepackage[utf8]{inputenc}
\usepackage[spanish]{babel}
\usepackage{geometry}
\usepackage{tikz}
\usepackage{tikz, pgfplots}
\pgfplotsset{compat=1.18}
\usepackage{xcolor}
\usepackage{amssymb}
%configuracion de la geometria de la pagina 
\geometry{a4paper, margin=1in}

\setlength{\parskip}{1em}

\title{\textbf{Resolucion de Limites y sus Indetermiaciones}}
\author{Zarza David}
\date{\today}


\begin{document}
\maketitle

%------ INICIO DEL EJEMPLO 1 -------
\section{Ejemplo 1: Indeterminacion $\frac{0}{0}$}

Consideremos el siguiente limite: 

$$\lim_{x \to 2^+} \frac{\sqrt{x^2 - 4 }}{x -2} = \frac{\sqrt{2-2}}{2^4-4}=\frac{0}{0}$$

Para resolver esta indeterminacion, podemos aplicar caso de factoreo, Ruffini, Multiplicar o dividir por el conjugado.
\subsection{Tipos de factoreo mas frecuentes en limites}

\begin{itemize}
    \item Factor comun: $6x^2 + 9x  = 3x(2x + 3)$ Sacamos el numero o letra que aparece en todos los terminos.
    \item Diferencia de cuadrados: \textbf{Reconocer $a^2 - b^2 = (a-b)(a+b)$} ejemplo, $x^2 - 9 = (x-3)(x+3) $, aqui $a = x,\ y \ b = 3 $.
    \item Trinomio cuadrado perfecto: \textbf{Reconocer $a^2 + 2ab + b^2 = (a+b)^2$} ejemplo, $x^2 + 6x + 9 = (x+3)^2$, aqui $a = x ,\ y \ b= 3$.
    \item Trinomio de la forma $ax^2 + bx + c$: \textbf{Buscas dos numeros que sumen b y multipliquen c}, ejemplo, $x^2 - 5x + 6 = (x-2)(x-3)$, porque $-2 + (-3) = -5 \ y \ (-2).(-3)=6$.
    \item Sacar un signo menos (cuando conviene): $-x^2 + 4x - 3 = \ -(x^2-4x+3)= \ -(x-1)(x-3)$. 
\end{itemize}
Continuando de resolver el ejemplo anterior 

$\lim_{x \to 2^+} \frac{\sqrt{x^2 - 4 }}{x -2} . \frac{\sqrt{x-2 }}{\sqrt{x -2}} = \frac{\sqrt{x-2}^2}{(x^2-4).\sqrt{x-2}}= $(aplicamos diferencia de cuadrados y cancelamos la raiz elevada al cuadrado)
$$ \lim_{x \to 2^+} \frac{x-2}{(x-2)(x+2)\sqrt{x-2}} = \lim_{x \to 2^+} \frac{1}{(2+2)\sqrt{2-2}} = \frac{1}{0} = \infty$$ 
\subsection{Operaciones con Infinito}
\begin{itemize}
    \item $\infty \pm k = \infty $
    \item $\infty . k = \infty $ si $k \neq 0 $
    \item $\frac{0}{k} = 0 $
    \item $\frac{k}{0} = \infty $
    \item $\frac{0}{\infty} = 0 $
    \item $\frac{k}{\infty} = 0$
    \item $\frac{\infty}{0} = \infty$ 
\end{itemize} 
\subsection{Limites con sen y tg}

\begin{itemize}
    \item $$\lim_{x \to 0} \frac{\sen x}{x} = 1$$ \ $$\sen x \sim x $$  \center el simbolo en medio de sen x y x significa que son equivalentes
    \item $$\lim_{x \to 0} \frac{\tan x}{x} = 1$$ \ $$\tan x \sim x $$ 
\end{itemize}
\begin{center}
$$\lim_{x \to 0} \frac{\sen^3 (2x) - x}{\tan^3(3x) + \sqrt{x}} =  $$ \ 
$$\lim_{x \to 0} \frac{(2x)^3 - x}{(3x)^2 + \sqrt{x}} = \frac{0}{0}$$  si aplicamos limite a esta funcion, obtenemos $\frac{0}{0}$, asi que trabajaremos sobre esto.
$$\lim_{x \to 0} \frac{8x^3 - x}{9x^2 + \sqrt{x}}$$   recordemos la operacion de dividir y Multiplicar por el conjugado 
$$\lim_{x \to 0} \frac{8x^3 - x}{9x^2 + \sqrt{x}} . \frac{ 9x^2 - \sqrt{x}}{9x^2 - \sqrt{x}} = \frac{(8x^3 - x). (9x^2 - \sqrt{x})}{(9x^2)^2 - \sqrt{x}^2}$$ cancelamos la raiz y escribimos el resultado del 9 elevado al cuadrado dos veces
$$\lim_{x \to 0} \frac{(8x^3 - x). (9x^2 - \sqrt{x})}{81x^4 - x} = \lim_{x \to 0} \frac{x(8x^2 - 1). (9x^2 - \sqrt{x})}{x(81x^3 - 1)}$$ simplificamos las x que sacamos como factor comun y aplicamos limite 
$$\lim_{x \to 0} \frac{(80^2x0 - 1). (9^2x0 - \sqrt{0})}{(81^3x0 - 1)} = \frac{0}{-1} = 0$$
\end{center}
\subsection{Limites de $\infty - \infty$}
$$ \lim_{x \to \infty} (\sqrt{x^2 +2} - x) = \infty - \infty $$
Para resolver este tipo de indeterminacion, multiplicamos y dividimos por el conjugado
$$ \lim_{x \to \infty } (\sqrt{x^2 +2} - x) . \frac{\sqrt{x^2+2}+x}{\sqrt{x^2+2}+x} $$
$$ \lim_{x \to \infty} \frac{\sqrt{x+2}^2-x^2}{\sqrt{x^2+2}+x} = \frac{\infty}{\infty}$$
$$\lim_{x \to \infty} \frac{x^2+2 - x^2}{\sqrt{x^2+2}+2} = \lim_{x \to \infty} \frac{2}{\sqrt{x^2+2}+2} = \frac{2}{\infty} = 0$$
\newpage
Pasamos a otro ejemplo del mismo tipo:
$$ \lim_{x \to 1} \frac{1}{x-1} - \frac{x}{x^2-1} = \infty - \infty \ $$
Lo primero es encontrar un comun denominador para las dos fracciones. Factorizamos la segunda ecuacion
$$ \lim_{x \to 1} \left(\frac{1}{x-1} - \frac{x}{(x-1)(x+1)}\right) $$ 
Ahora tenemos un comun denominador, que es $(x-1)(x+1)$, y podemos escribir la ecuacion como:
$$ \lim_{ x \to 1 } \left(\frac{1x(x+1)}{(x-1)(x+1)} - \frac{x}{(x-1)(x+1)} \right)$$
Al tener el mismo denominador, podemos restar los numeradores: 
$$ \lim_{x \to 1} \frac{x+1-x}{(x-1)(x+1) } \text \ remplazamos \ el \ limite \ y \ nos \ queda: \ \frac{1}{0}= \infty$$

Vamos a ver otro ejemplo pero mas complejo: 

$$ lim_{x \to \infty} \frac{\sqrt[3]{9x^8+6x^3-3}}{2x^3-x^2+9x} = \frac{\infty}{\infty}$$
Para resolver esta indeterminacion, dividimos el numerador y el denominador por la mayor potencia de x que aparezca, que en este caso es $x^3$
$$ \lim_{x \to \infty} \frac{\frac{\sqrt[3]{9x^3+6x^3-3}}{x^3}}{\frac{2x^3}{x^3}-\frac{x^2}{x^3}+\frac{9x}{x^3}}  $$
Un pequeño detalle es que, $\frac{x^2}{x^3} $ \ queda como $\frac{1}{x}$
Siguiendo con la simplificacion, cancalamos las x que podamos y luego nos queda: 
$$ \lim_{x \to \infty} \frac{\sqrt[3]{\frac{9x^3}{x^9}+\frac{6x^3}{x^9}-\frac{3}{x^9}}}{2-\frac{1}{x}-\frac{9}{x^2}}$$
$$ \lim_{x \to \infty} \frac{\sqrt{\frac{9}{x}+\frac{6}{x^6}-\frac{3}{x^9}}}{2-\frac{1}{x}+\frac{9}{x^2}} = \frac{0}{2} = 0$$

\section{Indeterminacion del tipo $1^\infty$}
Para resolver este tipo de indeterminaciones, necesitamos aplicar un metodo, el cual es el siguiente: 
$$ \lim_{x \to \infty} \left( 1+ \frac{1}{x} \right)^x = e $$
A continuacion, resolveremos un ejercicio: 
$$ lim_{x \to \infty} \left( 1- \frac{3}{5x}\right)^{2x-1} $$
Pasos algebraicos:
Divido por 3 y la resta la pasamos abajo: 

$$ \lim_{x \to \infty} \left( 1+ \frac{1}{-\frac{5x}{3}}\right)^{-\frac{5x}{3}.(-\frac{3}{5x}).(2x-1)}$$
Lo que hicimos fue elevar la base a la potencia porque necesitamos que sean iguales, y luego la multiplicamos por su inverso para que se cancele y no cambie nada de la expresion orginal.
\newpage
Vamos a aplicar una propiedad de potencias, $(a^p)^q = a^{p.q}$ 

$$ \lim_{x \to \infty} \left[\left(1+\frac{1}{-\frac{5x}{3}}^{-\frac{5x}{3}}\right)^{-\frac{3}{5x}.(2x-1)}\right] = $$

Como ya tenemos la base y la potencia iguales, podemos aplicar el limite: 

$$ e^{\lim_{x \to \infty}} \frac{-6x+3}{5x}= \frac{\infty}{\infty}$$

Recordemos que para resolver este tipo de indeterminaciones, utilizamos dividir todos los miembros por la mayor potencia de x que aparezca.

$$ e^{ \lim_{x \to \infty}} \frac{-6+\frac{3}{x}}{5} = e^{-\frac{6}{5}}$$

Vamos con otro caso del mismo tipo:

$$ \lim_{x \to \infty} \left(\frac{2x-1}{2x-3}\right)^{3x+2} $$

En este caso, vamos a sumar y restar 3 en el numerador para que nos quede una expresion de la forma $1 + \frac{1}{x}$ 

$$ \lim_{x \to \infty} \left(\frac{2x-3+3-1}{2x-3}\right)^{3x+2}$$

El siguiente paso es separar la fraccion en dos partes: 

$$ \lim_{x \to \infty} \left(\frac{2x-3}{2x-3} + \frac{2}{2x-3}\right)$$ 

Como se puede observar, podemos simplificar la primera parte de la fraccion, y nos queda 1 y en la segunda parte tendremos que dividir por 2 para hallar el 1 que nos falta.

$$ \lim_{ x \to \infty} \left(1+\frac{\frac{2}{2}}{\frac{2x-3}{2}}\right)^{3x+2} = \lim_{x \to \infty } \left(1+ \frac{1}{\frac{2x-3}{2}}\right)^{3x+2}$$

Ahora es solamente aplicar los pasos que ya vimos en el anterior ejemplo

$$ \lim_{x \to \infty} \left(1+ \frac{1}{\frac{2x-3}{2}}\right)^{\frac{2x-3}{2}.\frac{2}{2x-3}.(3x+2)} $$

$$ \lim_{x \to \infty} \left[\left(1+\frac{1}{\frac{2x-3}{2}}^{\frac{2x-3}{2}}\right)\right]^{\frac{6x+4}{2x-3}} $$

$$ e^{\lim_{x \to \infty} \frac{6x+4}{2x-3}} = \frac{\infty}{\infty} \ \  \ e^{\lim_{x \to \infty} \frac{6+\frac{4}{x}}{2-\frac{3}{x}}} = e^{\frac{6}{2}} = e^3$$

\newpage
\begin{center}
    \title{\LARGE\color{red}\textbf{Asintotas y Continuidad}}
\end{center}
\section{Tipos de Asintotas}
\begin{itemize}
    \item Para hallar la asintota horizontal, utilizaremos: 
     $$  \lim_{X \to \infty} f(x) = k ; k \ \epsilon \Re  $$
    \item Para hallar la asintota vertical, utilizaremos:
    $$ \lim_{x \to a} f(x) = \infty$$
    \item Para hallar la asintota oblicua, utilizaremos:
    $$ y = px + q$$ $$ P = \lim_{x \to \infty} \frac{f(x)}{x}$$ $$ Q= \lim_{x \to p}\left(f(x)- Px\right) $$
\end{itemize}

Las asintotas verticales no estan definidas en un punto, las Asintotas horizontales tiene valor de la ordenada, otro dato
de las asintotas horizontal es que si existen, significa que no existe asintota oblicua.

$$ Hallar \ las \ asintotas \ (1a) \ f(x) = \frac{3-x}{x^2-9}$$
Analizamos A.H primero: 
$$ \lim_{x \to \infty} \frac{3-x}{x^2 - 9} = \frac{\infty}{\infty}$$
dividimos por la x de mayor potencia
$$ \lim_{x \to \infty} \frac{\frac{3}{x^2}-\frac{x}{x^2}}{\frac{x^2}{x^2}-\frac{9x}{x^2}} = \lim_{x \to \infty}\frac{0-0}{1-0} = 0$$

Entonces tenemos nuestra A.H que se escribe, $ A.H: \ \ y=0$

Para las asintotas verticales, voy a buscar aquellos valores que anulen el denominador

Analisis de la A.V:
En este caso, lo que anula el denominador son el 3 y el -3. 

y se los llama como "Posibles Asintotas Verticales (A.V)".

$$ \lim_{x \to 3} \frac{3-x}{x^2-9} = \frac{0}{0}$$
$$ \lim_{x \to 3} \frac{-(x-3)}{(x-3)(x+3)}$$

Sacamos Factor comun -1 para poder cancelar la operacion x-3

$$ \lim_{x \to 3 } \frac{-1}{x+3} = \frac{-1}{6} $$

$\therefore$ $\nexists$ A.V en $x=3$ , si analizamos x = -3 obtendremos: $\therefore$$\exists$ A.V en x= -3

\newpage

\begin{center}
\begin{figure}[h!]
    \center
    \begin{tikzpicture}[scale=1.2]
    \begin{axis}[xmin=-5,xmax=5,ymin=-5,ymax=5,axis x line=center,axis y line=center,xlabel=$x$,ylabel=$y$]
        \addplot[red!80!white,thick,samples=200,domain=-3:3]{(3-x)/(x^2-9)};
    \end{axis}   
    \end{tikzpicture}
    \caption{Funci\'on $f(x) = \frac{3-x}{x^2-9}$}
\end{figure}
\end{center}

Analizaremos otra funcion: 

$$ f(x)= \frac{x^2+2}{x-2}$$

A.H:
$$ \lim_{x \to \infty} \frac{\frac{x^2}{x^2}+\frac{2x}{x^2}}{\frac{x}{x^2}-\frac{2x}{x^2}}= \frac{1+0}{0+0} = \infty$$
$\therefore$ \ $\nexists$ A.H

Entonces si existe A.O

Analizaremos la A.V:

$$ \lim_{x \to 2} \frac{x^2+2x}{x-2x} = \frac{8}{0} = \infty $$ $\therefore$ \ $\exists$ A.V en x = 2

Analizamos la A.O:

$$ P = \lim_{x \to \infty} \frac{\frac{x^2+2x}{x-2}}{\frac{x}{1}} $$
realizamos extremo por extremo, nos quedara: 
$$ \lim_{x \to \infty} \frac{x^2+2x}{x^2-2x} = \frac{\infty}{\infty}$$

Un dato muy importante es que cuando tenemos el numerador y el denominador del mismo grado 
el limite nos dara"1"
$P=1$

$$Q= \lim_{x \to \infty} \left(\frac{x^2+2x}{x-2}-1.x\right)= \lim_{x \to \infty} \frac{x^2+2x-x^2+2x}{x-2}= \lim_{x \to \infty} \frac{4x}{x-2} = \frac{\infty}{\infty}$$

Salvando la indeterminacion,como son del mismo grado, nos quedara:
$$ \lim_{x \to \infty} \frac{\frac{4x}{x}}{\frac{x}{x}-\frac{2}{x}} = \lim_{x \to \infty} = 4$$ 

\begin{center}
\begin{figure}[h!]
    \center
    \begin{tikzpicture}[scale=1.2]
    \begin{axis}[xmin=-10,xmax=10,ymin=-10,ymax=20,axis x line=center,axis y line=center,xlabel=$x$,ylabel=$y$]
        \addplot[red!80!white,thick,samples=200,domain=-3:1.9]{(x^2+2*x)/(x-2)};
        \addplot[red!80!white,thick,samples=200,domain=2.1:3]{(x^2+2*x)/(x-2)};
    \end{axis}   
    \end{tikzpicture}
    \caption{Funci\'on $f(x) = \frac{x^2+2x}{x-2}$}
\end{figure}
\end{center}


\end{document}